\chapter{
پیاده‌سازی نگاشت تصادفی
}

در این فصل به بررسی نحوه پیاده‌سازی نگاشت تصادفی و نحوه بررسی عملکرد آن می‌پردازیم همان‌طور که در فصل دوم 
% TODO reference chapter 2 part pca
اشاره شد تصویر تصادفی به عنوان یک روش کاهش بعد باید با دیگر روش‌های کاهش بعد مقایسه شود. در همین راستا مهمترین موردی که باید بدان توجه کرد مقایسه با روش‌های مرسوم کاهش بعد مانند
\lr{PCA}
است.

برای پیاده‌سازی روش نگاشت تصادفی مجموعه داده‌های مختلفی تهیه‌ و بر روی آن‌ها این روش برای کاهش بعد به دو بعد و سه بعد اعمال شده است. نتایج پس از خوشه‌بندی مورد مقایسه با نتایج خوشه‌بندی بدون کاهش بعد شده است. این فرآیند درکی از حوزه کاربرد نگاشت تصادفی و عملکرد آن بر روی داده‌های واقعی به ما می‌دهد.


\section{
داده‌‌های مورد استفاده
}

در این پایان‌نامه از ۷ مجموعه‌داده‌ی استاندارد و پرکاربرد، آیریس، اسکناس، دیابت، تیروئید، خرچنگ، بذر، پروتئین استفاده شده است.

دادگا‌ن‌های آیریس، بذر و پروتئین از سایت 
\lr{https://archive.ics.uci.edu/ml/index.php}
گردآوری شده و سایر مجموعه داده‌ها در کتابخانه‌های نرم‌افزار 
\lr{\textit{R}}
وجود دارد. یکی از ویژگی‌های داده‌های دم‌سنگین این است که شامل نقاط دورافتاده باشند. نقاط دور افتاده داد‌ها را بررسی کردیم. داده‌های اسکناس، آیریس، بذر، پروتئین، تیروئید، دیابت، خرچنگ دارای به ترتیب ۲۰، ۵۰، ۷۷، ۱۰۳، ۵۵، ۴۴، ۱۴ نقطه دور افتاده هستند. متاسفانه آزمونی برای برازش توزیع پایدار چند متغییره وجود ندارد و فقط می‌توانیم آزمون برازش یک متغیره توزیع پایدار را بررسی کنیم و مشاهده کردیم هیچ کدام از متغییر‌های دادگان‌های خرچنگ، اسکناس و بذر به توزیع پایدار برازش نشدند اما برخی از متغیرهای دادگان‌های تیروئید، آیریس، دیابت و پروتئین به توزیع پایدار برازش شدند. همچنین هیچ کدام از دادگان‌ها به توزیع نرمال چند متغیره برازش نشد و آزمون نرمال بودن رد شد. توزیع این داده‌ها را بررسی کردیم و با استفاده از چهار آزمون ماردیا، هنس-زیرکلر، شاپیرو و رویستون، نرمال بودن این دادگان‌ها رد شده است.

\subsection{
داده‌های غده تیروئید
}

این مجموعه داده که با نام اختصاری 
\lr{Thyroid}
در جداول مشخص شده است، از کتابخانه
\lr{mclust}
\cite{rmclust}
در 
\lr{\textit{R}}
استخراج شده. داده‌ها شامل پنج تست آزمایشگاهی انجام شده بر روی 215 بیمار است. داده‌های بدین منظور مورد استفاده قرار گرفته‌اند تا بتوانیم پیش‌بینی کنیم که آیا غده تیروئید بیمار را می‌توان در دسته‌ی «مواردی که غده تیروئید به شکل نرمال کار می‌کند»
\LTRfootnote{euthyroidism}
، «غده تیروئید کم کار است و هورمون کافی تولید نمی‌کند»
\LTRfootnote{hypothyroidism}
یا «غده تیروئید بیش فعال است و مقدار زیادتری هورمون 
\lr{\textbf{T3}}
و یا
\lr{\textbf{T4}}
\LTRfootnote{thyroxine}
ترشح می‌کند»
\LTRfootnote{hyperthyroidism}
. درمان عملکرد تیروئید بر اساس اطلاعات کامل پزشکی صورت گرفته است. شامل، سابقه‌ی درمانی، اسکن، ...

داده به شکل یک قاب دا‌ده‌ها
\LTRfootnote{data.frame}
شامل متغییر‌های زیر در 
\lr{\textit{R}}
قرار می‌گیرند:
\begin{itemize}
\item
\lr{\textbf{Diagnosis}}
درمانی که برای تیروئید تجویز شده است. شامل:
\lr{Hyper, Normal, Hypo}
\item
\lr{\textbf{RT3U}}
تست میزان جذب 
\lr{\textbf{T3-resin}}
(به شکل درصد)
\item
\lr{\textbf{T4}}
کل سرم تیروکسین که با روش جایگزاری ایزوتوپ
\LTRfootnote{isotopic displacement method}
اندازه‌گیری شده است.
\item
\lr{\textbf{‌T3}}
کل سرم تیرودوتیروئین 
\LTRfootnote{triiodothyronine}
که با روش رادیو ایمونو
\LTRfootnote{radioimmuno}
اندازه‌گیری شده است.
\item
\lr{\textbf{TSH}}
هورمون اصلی محرک تیروئید که با روش رادیوایمونو اندازه‌گیری شده است.
\lr{\textbf{DTSH}}
مقدار حداکثر اختلاف
\lr{\textbf{TSH}}
بعد از تزریق
$200 \mu gr$
هورمون رها ساز تیروتروپین
\LTRfootnote{thyrotropin}
در مقایسه با مقدار پایه.
\end{itemize}


\subsection{
داده‌های دیابت
}

برای این مجموعه داده که به اختصار 
\lr{Diabetes}
در جداول ذکر شده است، از کتابخانه
\lr{mclust}
\cite{rmclust}
نرم‌افزار 
\lr{\textit{R}}
استفاده شده است.

این مجموعه داده شامل سه اندازه‌گیری است که بر روی 145 بیمار غیر چاق صورت گرفته است. این بیماران به سه گروه تقسیم شده‌اند.

داده‌ها به شکل یک قاب داده‌ها شامل متغییرهای زیر در 
\lr{\textit{R}}
قرار دارند:

\begin{itemize}
\item
\lr{\textbf{class}}
نوع دیابت را مشخص می‌کند و به رده‌ی 
\lr{Normal, Overt, Chemical}
طبقه‌بندی می‌شود.
\item
\lr{\textbf{glucose}}
مساحت زیر نمودار گلوکوز در پلاسما پس از سه ساعت آزمون 
\lr{\textbf{OGTT}}
\LTRfootnote{oral glucose tolerance test}
\item
\lr{\textbf{insulin}}
مساحت زیر نمودار انسولین در پلاسما پس از سه ساعت آزمون
\lr{OGTT}
\item
\lr{\textbf{sspg}}
میزان گلوکوز پلاسما در حالت پایدار
\end{itemize}

\subsection{
داده‌های خرچنگ‌ها
}

این مجموعه داده‌ها اندازه‌گیری موفولوژیکی
\LTRfootnote{morphological}
بر روی خرچنگ لپتوکراپسوس  که به اختصار
\lr{Crabs}
در جداول نشان داده شده‌اند، از کتابخانه‌ی 
\lr{MASS}
\cite{rMASS}
در نرم‌افزار 
\lr{\textit{R}}
استخراج شده‌اند.

قالب داده‌های خرچنگ‌ها شامل 200 سطر و هشت ستون است که بیانگر 5 اندازه‌ی مورفولوژیک برای 50 خرچنگ از دو رنگ و دو جنسیت است.

قالب داده‌ی مربوط شامل این ستون‌ها است:

\begin{itemize}
\item
\lr{\textbf{sp}}
گونه‌ی «\lr{B}» برای رنگ آبی و گونه‌ی «\lr{O}» برای رنگ نارنجی
\item
\lr{\textbf{sex}}
جنسیت
\item
\lr{\textbf{index}}
اندیس از 1 تا 50 در داخل هر گروه
\item
\lr{\textbf{FL}}
لوب پیشانی بر حسب میلیمتر
\item
\lr{\textbf{RW}}
عرض عقب بر حسب میلیمتر
\item
\lr{\textbf{CL}}
طول لاک بر حسب میلیمتر
\item
\lr{\textbf{CW}}
عرض لاک بر حسب میلیمتر
\item
\lr{\textbf{BD}}
عمق بدن بر حسب میلیمتر
\end{itemize}

\subsection{
داده‌های اسکناس سوئیس
}

این داده‌ها که به اختصار «
\lr{Swiss banknotes}
» در جداول بیان شده است از کتابخانه‌ی 
\lr{mclust}
\cite{rmclust}
در 
\lr{\textit{R}}
استخراج شده‌اند.

این داده‌ها شامل مجموعه‌ی ۶ اندازه‌گیری مربط به ۱۰۰ اسکناس اصل و ۱۰۰ اسکناس تقلبی ۱۰۰۰ فرانکی قدیمی است. قالب داده‌های مربوط شامل موارد زیر است:

\begin{itemize}
\item
\lr{\textbf{Status}}
وضعیت اسکناس که شامل دو حالت اصل (%
\lr{genuine}%
)
یا تقلبی (%
counterfeit%
) می‌شود.
\item
\lr{\textbf{Length}}
طول اسکناس بر حسب میلیمتر
\item
\lr{\textbf{Left}}
عرض لبه‌ی چپ بر حسب میلیمتر
\item
\lr{\textbf{Right}}
عرض لبه‌ی راست بر حسب میلیمتر
\item
\lr{\textbf{Bottom}}
عرض حاشیه پایین بر حسب میلیمتر
\item
\lr{\textbf{Top}}
عرض حاشیه بالا بر حسب میلیمتر
\item
\lr{\textbf{Diagonal}}
طول قطر بر حسب میلیمتر
\end{itemize}

\subsection{
داده‌های آیریس
}

این مجموعه داده که در جداول به اختصار 
\lr{Iris}
نامیده می‌شود از سایت 
\lr{UCI}
\cite{uci_iris}
تهیه شده است.

این مجموعه داده‌ها احتمالا شناخته‌شده‌ترین مجموعه داده در ادبیات شناخت الگور%
\LTRfootnote{pattern recognition}
است. مقاله فیشر که یک مقاله کلاسیک در این حوزه است و بارها مورد ارجاع قرار گرفته از این مجموعه داده استفاده می‌کند. این مجموعه داده شامل سه رده‌ی ۵۰ تایی است. که هر رده نمیونه‌ای از یک نوع گیاه آیریس است. یک رده که از دو گروه دیگر به طور خطی قابل تفکیک است ولی دو گروه بعدی به طور خطی قابل تفکیک نیستند.

داده‌ها در قالب یک فایل متنی با ستون‌های زیر از سایت 
\lr{UCI}
دانلود شده‌اند:

\begin{itemize}
\item
\lr{\textbf{Sepal length}}
طول کاسبرگ که به سانتی‌متر بیان شده است.
\item
\lr{\textbf{Sepal Width}}
عرض کاسبرگ که به سانتی‌متر بیان شده است.
\item
\lr{\textbf{Petal length}}
طول گلبرگ که به سانتی‌متر بیان شده است.
\item
\lr{\textbf{Petal width}}
عرض گلبرگ که به سانتی‌متر بیان شده است.
\item
\lr{\textbf{Class}}
نوع گل آیریس که می‌تواند یکی از سه مقدار 
\lr{setosa, versicolor, virginica}
را داشته باشد.
\end{itemize}

\subsection{
مجموعه داده‌های دانه‌ها
}

این مجموعه داده که با مخفف «%
\lr{Seeds}%
» در جداول و نمودارها آورده شده است، از سایت 
\lr{UCI}
\cite{uci_seeds}
جمع‌آوری شده است.

گروه مورد آزمایش شامل سه گروه از گونه‌ّای مختلف گندم است. گونه‌ی کاما، رز و کانادایی%
\LTRfootnote{Kama, Rose, Canadian}%
. ۷۰ مورد از هر گونه به طور تصادفی برای آزمایش انتخاب شده‌اند. تصاویری با کیفیت بالا از ساختار دورنی دانه‌گندم با استفاده از تکنیک پرتو ایکس نرم%
\LTRfootnote{saft x-ray}
تهیه شده است. این روش یک روش غیر مخرب است و نسبت به دیگر روش‌های پیچیده‌ی تصویربرداری مانند میکروسکوپ روبشی و تکنولوژی لیزر، ارزانتر است. تصاویر بر روی فیلم پرتو ایکس کداک ۱۳ در ۱۸ سانتی‌متر ضبط شده است. مطالعه براساس دانه‌ی جمع‌آوری شده از مزارع آزمایشی به طور ترکیبی انجام شده است و در انستیتو آکروفیزیکی آکادمی لهستان در لوبین مورد بررسی قرار گرفته است.%
\LTRfootnote{institute of agrophysics of the Polish academy of science in Lubin}
این داده‌ها می‌تواند برای رده‌بندی و خوشه‌بندی مورد استفاده قرار گیرد.

این مجموعه داده در قالب یک فایل متنی شامل ۸ ستون تهیه شده است. توضیح ستون‌ها شامل موارد زیر است:

\begin{itemize}
\item
\lr{\textbf{A}}
مساحت
\item
\lr{\textbf{P}}
محیط
\item
\lr{\textbf{C}}
فشردگی 
$4 \pi A / P^2$
\item
\lr{\textbf{length of kernel}}
طول هسته
\item
\lr{\textbf{width of kernel}}
عرض هسته
\item
\lr{\textbf{symmetry coefficient}}
ضریب تقارن
\item
\lr{\textbf{length of kernel groove}}
طول رشد هسته
\item
\lr{\textbf{Class}}
نوع دانه
\end{itemize}

\subsection{
مجموعه داده‌ی بیان پروتئین موش
}

این مجموعه داده که با نام اختصاری 
\lr{MPE}
\LTRfootnote{mice protein expression}
در جداول و نمودارها آورده شده است، از وب سایت
\lr{UCI}
\cite{uci_MPE}
جمع‌آوری شده است. مجموعه داده شامل سطوح بیان ۷۷ تغییرات پروتئین/پروتئین‌ها است که سیگنال قابل اندازه‌گیری در شکست پوسته‌ی هسته تولید می‌کنند.

۳۳ موش کنترلی و ۳۴ موش تیروزومی
\LTRfootnote{trisomic}
(دارای سندرم داون) مورد مطالعه قرار گرفته‌اند. که در کل شامل ۷۲ موش می‌شود. در این آزمایش، ۱۵ اندازه‌گیری مربوط به هر پروتئین برای هر موش اندازه‌گیری شده است. بنابراین برای موش‌های کنترلی، 
$38 \times 15 = 570$
اندازه‌گیری و برای موش‌های تیروزومی 
$34 \times 15 = 510$
اندازه‌گیری انجام شده است. این مجموعه‌داده شامل مجموعا ۱۰۸۰ اندازه‌گیری برای هر پروتئین است. هر اندازه‌گیری می‌تواند به عنوان یک نمونه/موش مستقل در نظر گرفته شود.

۸ رده‌ی موش‌ها بر اساس مشخصاتی مانند ژنوتیپ
\LTRfootnote{genotype}
، رفتار و درمان طبقه‌بندی شده‌اند. از نظر ژنوتیپ موش‌ها می‌توانند کنترل با تیروزومی باشند. از نظر رفتاری، به یادگیری تحریک شده‌اند (زمینه-شوک)
\LTRfootnote{context-shock}
 و بقیه تحریک نشده‌اند (شوک-زمینه) و برای سنجش تأثیر دارو، داروی ممانتین%
\LTRfootnote{memantine}
قابلیت یادگیری در موش‌های تیروزومی را افزایش می‌دهد. به برخی دارو تزریق شده است و به بقیه خیر.

\textbf{
رده‌ها:
}

\begin{itemize}
\item
\lr{\textbf{c-CS-s}}
: موش کنترلی، تحریک شده به آموزش، تزریق با محلول آب و نمک (۹ موش)
\item
\lr{\textbf{c-CS-m}}
: موش کنترلی، تحریک شده به آموزش، تزریق با محلول ممانتین (۱۰ موش)
\item
\lr{\textbf{c-SC-s}}
: موش کنترلی، تحریک نشده به آموزش، تزریق با محلول آب و نمک (۹ موش)
\item
\lr{\textbf{c-SC-m}}
: موش کنترلی، تحریک نشده به آموزش، تزریق با محلول ممانتین (۱۰ موش)
\item
\lr{\textbf{t-CS-s}}
: موش تیروزومی، تحریک شده به آموزش، تزریق با محلول آب و نمک (۷ موش)
\item
\lr{\textbf{t-CS-m}}
: موش تیروزومی، تحریک شده به آموزش، تزریق با محلول ممانتین (۹ موش)
\item
\lr{\textbf{t-SC-s}}
: موش تیروزومی، تحریک نشده به آموزش، تزریق با محلول آب و نمک (۹ موش)
\item
\lr{\textbf{t-SC-m}}
: موش تیروزومی، تحریک نشده به آموزش، تزریق با محلول ممانتین (۹ موش)
\end{itemize}

هدف تعیین زیر‌گروهی از پروتئین‌ها است که بین رده‌ها تمایز ایجاد می‌کنند.

داده‌ها در قالب یک فایل متنی شامل ستون‌های زیر تهیه شده‌اند:

\begin{itemize}
\item
۱: شناسه‌ی موش
\item
۲ تا ۷۸: میزان بیان پروتئین برای ۷۷ پروتئین
\item
۸۰: نحوه‌ی درمان: ممانتین(%
\lr{m}%
) یا نمک(%
\lr{s}%
)
\item
۸۱: رفتار (%
\lr{CS}
یا 
\lr{SC}%
)
\item
۸۲: رده‌ها
\end{itemize}





\section{
الگوریتم پیاده‌سازی نگاشت تصادفی
}

پیاده‌سازی نگاشت تصادفی نسبتا ساده‌است. چون در همه حالات درایه‌های ماتریس تصادفی 
$\mathbf{R} \in \mathbb{R}^{D \times k}$
به طور 
\lr{i.i.d}
انتخاب می‌شوند. چالشی از جهت وابستگی متغییر‌های تصادفی در طول فرآیند تولید نداریم. به طور کلی از دو نوع توزیع جهت تولید ماتریس تصادفی 
$\mathbf{R}$
استفاده کردیم. یکی توزیع پایدار و دیگری توزیع زیرگاوسی بسیار تُنُک. توزیع‌های نرمال و کوچی حالت خاصی از توزیع نرمال قلمداد شده‌اند.
کد مربوط به پیاده‌سازی تصویر تصادفی در حالت پایدار به سادگی زیر است:

\begin{latin}
\begin{Verbatim}[baselinestretch=0.75]
randomProjection <- function(d, targetNumDimention, alpha = 2) {
  
  pm <- matrix(rstable(ncol(d) * targetNumDimention, alpha = alpha, beta = 0),
               nrow = ncol(d))
  
  d %*% pm
  
}
\end{Verbatim}
\end{latin}

این تابع که دو یا سه ورودی می‌تواند داشته باشد ماتریس داده 
$\mathbf{A} \in \mathbb{R}^{n \times D}$
را در قالب متغییر 
\verb|d|%
، تعداد ابعاد مد نظر برای  کاهش بعد
$k$
در قالب متغییر
\verb|targetNumDimention|
و پارامتر 
$\alpha$
توزیع پایدار با متغییر
\verb|alpha|
بیان شده است. پارامتر 
$\alpha$
می‌تواند در وردی‌های تابع ظاهر نشود و مقدار پیش‌فرض ۲ را دارد.

برای حالت تُنُک تابع 
\verb|sRandomProjection|
به شکل زیر پیاده‌سازی شده است.

\begin{latin}
\begin{Verbatim}[baselinestretch=0.75]
sRandomProjection <- function(d, targetNumDimention, s = 2) {
  
  r <- runif(ncol(d) * targetNumDimention, min = 0, max = 2*s)
  r[r <= 1] = -1
  r[r >= (2*s-1)] = 1
  r[r != 1 & r != -1] <- 0
  pm <- matrix(r,
               nrow = ncol(d))
  
  d %*% pm
  
}
\end{Verbatim}
\end{latin}

دو متغییر ورودی 
\verb|d|
و
\verb|targetNumDimention|
مانند تابع قبلی بیانگر داده‌هایی ورودی و تعداد ابعاد هدف است. ورودی
\verb|s|
ولی نشان دهنده پارامتر
$s$
در توزیع ورودی مربوط به این تابع است. در این حالت از یک توزیع گسسته‌ ما مقادیر 
$\{-1, 0, 1\}$
با احتمالات
$\{ \frac{1}{2s}, 1-\frac{1}{s}, \frac{1}{2s} \}$
استفاده می‌کنیم. هرچه مقدار پارامتر 
$s$
در توزیع بیشتر باشد بهتر است.



\section{
ارزیابی روش‌های کاهش بعد
}

برای ارزیابی روش‌های کاهش‌بعد، معیا‌ر‌های مختلفی معرفی شده است.%
\cite{gracia2014methodology}
یکی از روش‌های ارزیابی عملکرد کاهش بعد، خوشه‌بندی است. یکی دیگر از کاربردهای خوشه‌بندی علاوه بر کاهش بعد، ارزیابی کاهش بعد نیز است. برای ارزیابی روش‌های کاهش بعد با استفاده از خوشه‌بندی، ابتدا خوشه‌بندی را روی داده‌ها با تمام متغیرها انجام می‌دهیم و با استفاده از یکی از معیار ارزیابی خوشه‌بندی شاخص رند تعدیل شده که در بخش 
\ref{sec:ARI}
بیان شده است، عملکرد خوشه‌بندی را محاسبه می‌کنیم. سپس روش کاهش بعد را روی داده‌ها اعمال کرده و مجدد روی مجموعه داده‌های کاهش بعد یافته، خوشه‌بندی را پیاده‌سازی و عملکرد خوشه‌بندی را روی داده‌ها با ابعاد کمتر محاسبه می‌کنیم. سرانجام با مقایسه عملکرد خوشه‌بندی قبل و بعد از کاهش بعد، روش کاهش بعد را ارزیابی خواهیم کرد.

برای مقایسه روش‌های کاهش بعد بر اساس نگاشت تصادفی‌های متفاوت، از معیار نرخ خطار رده‌بندی استفاده می‌کنیم. معیار نرخ خطای رده‌بندی به صورت زیر تعریف می‌شود:

\begin{align}
C_e = 100 ( \mathrm{ARI}_d - \mathrm{ARI}_p )
\label{eq:1jT}
\end{align}

که در آن 
$\mathrm{ARI}_d$
و 
‌$\mathrm{ARI}_p$
به ترتیب شاخص رند تعدیل شده در فضای 
$\mathbb{R}^d$
و 
$\mathbb{R}^p$
می‌باشند
$(d < p)$
. 
$\mathrm{ARI}_d$
، شاخص رند تعدیل شده خوشه‌بندی با متغیرهای کمتر و 
$\mathrm{ARI}_p$
، شاخص رند تعدیل شده خوشه‌بندی با متغییرهای کامل است. هر چقدر معیار
$C_e$
بزرگتر باشد، بیانگر این است که کاهش بعد بهتری صورت گرفته است.
در جداول فصل بعد کاهش بعد به دو و سه بعد انجام شده است. و عملکرد کاهش بعد با معیار $C_e$ سنجیده شده است.



