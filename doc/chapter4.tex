\chapter{
پیاده‌سازی نگاشت تصادفی
}

در این فصل به بررسی نحوه پیاده‌سازی نگاشت تصادفی و نحوه بررسی عملکرد آن می‌پردازیم همان‌طور که در فصل دوم 
% TODO reference chapter 2 part pca
اشاره شد تصویر تصادفی به عنوان یک روش کاهش بعد باید با دیگر روش‌های کاهش بعد مقایسه شود. در همین راستا مهمترین موردی که باید بدان توجه کرد مقایسه با روش‌های مرسوم کاهش بعد مانند
\lr{PCA}
است.

برای پیاده‌سازی روش نگاشت تصادفی مجموعه داده‌های مختلفی تهیه‌ و بر روی آن‌ها این روش برای کاهش بعد به دو بعد و سه بعد اعمال شده است. نتایج پس از خوشه‌بندی مورد مقایسه با نتایج خوشه‌بندی بدون کاهش بعد شده است. این فرآیند درکی از حوزه کاربرد نگاشت تصادفی و عملکرد آن بر روی داده‌های واقعی به ما می‌دهد.

\section{
ارزیابی روش‌های کاهش بعد
}

برای ارزیابی روش‌های کاهش‌بعد، معیا‌ر‌های مختلفی معرفی شده است.%
\cite{gracia2014methodology}
یکی از روش‌های ارزیابی عملکرد کاهش بعد، خوشه‌بندی است. یکی دیگر از کاربردهای خوشه‌بندی علاوه بر کاهش بعد، ارزیابی کاهش بعد نیز است. برای ارزیابی روش‌های کاهش بعد با استفاده از خوشه‌بندی، ابتدا خوشه‌بندی را روی داده‌ها با تمام متغیرها انجام می‌دهیم و با استافده از یکی از معیارهای ارزیابی خوشه‌بند که در بخش 
% TODO ref chapter 2 ARI
بیان شده است، عملکرد خوشه‌بندی را محاسبه می‌کنیم. سپش روش کاهش بعد را روی داده‌ها اعمال کرده و مجدد روی مجموعه داده‌های کاهش بعد شده خوشه‌بندی را پیاده‌سازی و عملکرد خوشه‌بندی را روی داده‌ها با متغییر‌های کمتر محاسبه می‌کنیم. سرانجام با مقایسه عملکرد خوشه‌بندی قبل و بعد از کاهش بعد، روش کاهش بعد را ارزیابی خواهیم کرد.

\section{
داده‌‌های مورد استفاده
}

در این پایان‌نامه از ۷ مجموعه‌داده‌ی استاندارد و پرکاربرد، آیریس، اسکناس، دیابت، تیروئید، خرچنگ، بذر، پروتئین استفاده شده است.

دادگا‌ن‌های آیریس، بذر و پروتئین از سایت 
\lr{https://archive.ics.uci.edu/ml/index.php}
گردآوری شده و سایر مجموعه داده‌ها در کتابخانه‌های نرم‌افزار 
\lr{\textbf{R}}
وجود دارد. یکی از ویژگی‌های داده‌های دم‌سنگین این است که شامل نقاط دورافتاده باشند. نقاط دور افتاده داد‌ها را بررسی کردیم. داده‌های اسکناس، آیریس، بذر، پروتئین، تیروئید، دیابت، خرچنگ دارای به ترتیب ۲۰، ۵۰، ۷۷، ۱۰۳، ۵۵، ۴۴، ۱۴ نقطه دور افتاده هستند. متاسفانه آزمونی برای برازش توزیع پایدار چند متغییره وجود ندارد و فقط می‌توانیم آزمون برازش یک متغیره توزیع پایدار را بررسی کنیم و مشاهده کردیم هیچ کدام از متغییر‌های دادگان‌های خرچنگ، اسکناس و بذر به توزیع پایدار برازش نشدند اما برخی از متغیرهای دادگان‌های تیروئید، آیریس، دیابت و پروتئین به توزیع پایدار برازش شدند. همچنین هیچ کدام از دادگان‌ها به توزیع نرمال چند متغیره برازش نشد و آزمون نرمال بودن رد شد. توزیع این داده‌ها را بررسی کردیم و با استفاده از چهار آزمون ماردیا، هنس-زیرکلر، شاپیرو و رویستون، نرمال بودن این دادگان‌ها رد شده است.

\subsection{
اطلاعات غده تیروئید
}

این مجموعه داده که با نام اختصاری 
\lr{Thyroid}
در جداول مشخص شده است، از کتابخانه
\lr{mclust}
\cite{rmclust}
در 
\lr{\textbf{R}}
استخراج شده. داده‌ها شامل پنج تست آزمایشگاهی انجام شده بر روی 215 بیمار است. داده‌های بدین منظور مورد استفاده قرار گرفته‌اند تا بتوانیم پیش‌بینی کنیم که آیا غده تیروئید بیمار را می‌توان در دسته‌ی «مواردی که غده تیروئید به شکل نرمال کار می‌کند»
\LTRfootnote{euthyroidism}
، «غده تیروئید کم کار است و هورمون کافی تولید نمی‌کند»
\LTRfootnote{hypothyroidism}
یا «غده تیروئید بیش فعال است و مقدار زیادتری هورمون 
\lr{T3}
و یا
\lr{T4}
\LTRfootnote{thyroxine}
ترشح می‌کند»
\LTRfootnote{hyperthyroidism}
. درمان عملکرد تیروئید بر اساس اطلاعات کامل پزشکی صورت گرفته است. شامل، سابقه‌ی درمانی، اسکن، ...

داده به شکل یک قاب دا‌ده‌ها
\LTRfootnote{data.frame}
شامل متغییر‌های زیر در 
\lr{R}
قرار می‌گیرند:
\begin{itemize}
\item
\lr{Diagnosis}
درمانی که برای تیروئید تجویز شده است. شامل:
\lr{Hyper, Normal, Hypo}
\item
\lr{RT3U}
تست میزان جذب 
\lr{T3-resin}
(به شکل درصد)
\item
\lr{T4}
کل سرم تیروکسین که با روش جایگزاری ایزوتوپ
\LTRfootnote{isotopic displacement method}
اندازه‌گیری شده است.
\item
\lr{‌T3}
کل سرم تیرودوتیروئین 
\LTRfootnote{triiodothyronine}
که با روش رادیو ایمونو
\LTRfootnote{radioimmuno}
اندازه‌گیری شده است.
\item
\lr{TSH}
\end{itemize}
