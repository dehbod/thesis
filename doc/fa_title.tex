%% -!TEX root = AUTthesis.tex
% در این فایل، عنوان پایان‌نامه، مشخصات خود، متن تقدیمی‌، ستایش، سپاس‌گزاری و چکیده پایان‌نامه را به فارسی، وارد کنید.
% توجه داشته باشید که جدول حاوی مشخصات پروژه/پایان‌نامه/رساله و همچنین، مشخصات داخل آن، به طور خودکار، درج می‌شود.
%%%%%%%%%%%%%%%%%%%%%%%%%%%%%%%%%%%%
% دانشکده، آموزشکده و یا پژوهشکده  خود را وارد کنید
\faculty{دانشکده ریاضی و علوم کامپیوتر}
% گرایش و گروه آموزشی خود را وارد کنید
\department{
علوم کامپیوتر گرایش سیستم‌های کامپیوتری}
% عنوان پایان‌نامه را وارد کنید
\fatitle{
کاهش بعد داده‌های بزرگ مقیاس با استفاده از \\
نگاشت تصادفی
}
% نام استاد(ان) راهنما را وارد کنید
\firstsupervisor{دکتر عادل محمدپور}
%\secondsupervisor{استاد راهنمای دوم}
% نام استاد(دان) مشاور را وارد کنید. چنانچه استاد مشاور ندارید، دستور پایین را غیرفعال کنید.
\firstadvisor{دکتر هادی زارع}
%\secondadvisor{استاد مشاور دوم}
% نام نویسنده را وارد کنید
\name{ سیامک}
% نام خانوادگی نویسنده را وارد کنید
\surname{ دهبد}
%%%%%%%%%%%%%%%%%%%%%%%%%%%%%%%%%%
\thesisdate{دی 1397}

% چکیده پایان‌نامه را وارد کنید
\fa-abstract{
با ظهور داده‌های بزرگ مقیاس و دشواری در نگهداری و پردازش این داده‌ها در حافظه، مسئله کاهش بعد اهمیت زیادی پیدا کرده است. یکی از روش‌های کاهش بعد، نگاشت تصادفی است که می‌تواند بر روی مه‌داد‌هایی که بزرگ مقیاس هستند و همچنین بر روی جریان‌ داده‌ها، اعمال شود. مبنای این روش ضرب ماتریسی داده‌های اولیه در یک ماتریس تصویرگر است که بعد داده‌های اولیه را کاهش داده ولی اطلاعات آماری مورد نیاز در داده‌های اولیه را با دقت مورد نیاز نگه می‌دارد.\\
داده‌های بزرگ مقیاس داده‌هایی هستند که تعداد پارامترهای مدل از تعداد مشاهدات بیشتر است.\\
روش تصویر تصادفی برای کاهش بعد داده‌های بزرگ مقیاس مزایای متعددی نسبت به روش‌های دیگر کاهش بعد دارد. از جمله سرعت بالا در پردازش، نیاز به حافظه محدود، قابل اعمال بر روی جریان داده و قابل اعمال در شرایطی که تعداد پارامتر‌ها از مشاهدات بیشتر است. در این پایان‌نامه این روش برای داده‌های بزرگ مقیاس با دیگر روش‌های کاهش بعد مقایسه شده است. همچنین توانایی این روش برای داده‌هایی با توزیع پایدار غیر نرمال با دیگر روش‌های کاهش بعد مقایسه شده است.\\
با مقایسه روش‌های مختلف به این نتیجه رسیدیم که برای داده‌های بزرگ مقیاس روش کاهش بعد تصویر تصادفی کارایی خوبی دارد و با توجه به هزینه محاسباتی در خیلی از موارد بهینه است. ولی در بیشتر موارد تکرار این روش برای رسیدن به یک حالت بهینه لازم است.\\
}


% کلمات کلیدی پایان‌نامه را وارد کنید
\keywords{کاهش بعد، نگاشت تصادفی، توزیع پایدار، داده‌های بزرگ مقیاس}



\AUTtitle
%%%%%%%%%%%%%%%%%%%%%%%%%%%%%%%%%%
